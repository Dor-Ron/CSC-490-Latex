\documentclass{article}

\usepackage{amsthm, amssymb, amsmath}
\usepackage[margin=1in]{geometry}

% used in maketitle
\title{Week 2 Assignment}
\author{Dor Rondel}

% nicer formatting b/w paragraphs
\setlength{\parskip}{6pt}

% set the style for the theorems; optional
\theoremstyle{definition}
% define new environments: thm, prop
\newtheorem{thm}{Theorem}
\newtheorem{prop}[thm]{Proposition}
\newtheorem{fact}[thm]{Fact}
\newtheorem*{ex*}{Example}

\begin{document}
\maketitle

% Problem 1
\begin{prop}
If $A$ is a set then $A\cup\emptyset=A$.
\end{prop} 

\begin{proof}
To show $A\cup\emptyset=A$, we show $A \cup\emptyset\subseteq A$ and $A \subseteq A \cup \emptyset$ - of course, given that $A$ is a set. 

To prove the former, we take an arbitrary element $x$ in $A \cup \emptyset$ and show $x \in A$. By definition of union, if $x \in A \cup \emptyset$ then either $x \in A$ or $x \in \emptyset$. But $\emptyset$ has no elements; hence if $x \in A \cup \emptyset$ then $x \in A$. Thus $A \subseteq A \cup \emptyset$. 

Now suppose $x$ is an arbitrary element of $A$. Then clearly $x \in A$ or $x \in \emptyset$, so if $x \in A$, then $x \in A \cup \emptyset$ by definition of union. Thus $A \subseteq A \cup \emptyset$. 

Hence we conclude $A = A \subseteq \emptyset$.
\end{proof}

% Problem 2
\begin{thm}
There exist infintely many primes.
\end{thm}

\begin{proof}
First note that primes do exist, so the following argument is not vacuous. 

Suppose there is only a finite number of primes, $n$. Then $n \in \mathbb{N}$ and $n \neq 0$. Let us list the primes in inceasing order: 
\begin{equation*}
p_1, p_2, p_3, ..., p_{n-1}, p_{n}
\end{equation*}.
Let $M = p_1, p_2, p_3, ..., p_{n-1}, p_{n} + 1$. Clearly $M > 1$ and $M > p_i$ for $i = 1,2,3, ... n$ so $M$ must not be on the list of primes. By the Fundamental Theorom of Arithmetic (I), every positive integer is either $1$, prime, or composite, so $M$ must be composite. By the Fundamental Theorem of Arithmetic (II), every positive integer must have prime divisors so $M$ must have at least $1$ prime divisor. Thus there exists $i \in \{1,2,...n\}$ such that $p_i \vert M$ and there exists $y
\in \mathbb{Z}$ such that $M = p_iz$. Then % I think it should be $z \in \mathbb{Z}$ but handout said $y \in \mathbb{Z}$
\begin{equation*}
1 = M - p_1p_2...p_{n-1}p{n} = p_{iz} - p_1p_2...p_{i-1}p_ip_{i+1}...p_n
\end{equation*}
and so $1=p_i(z-p_1p_2...p_{i-1}p_{i+1}...p_n)$, implying $p_i \vert 1$. This is impossible! Therefore our assumption from start is false: There cannot exist a finite number of primes. We must conclude that there exists infinitely many primes.
\end{proof}

% Problem 3
\begin{fact}
The geometric series
\end{fact}
\begin{equation*}
\sum_{n=0}^{\infty}ar^{n-1} = a + ar + ar^2 + ...
\end{equation*}
is convergent with sum $\frac{a}{1-r}$ when $|r| < 1$. If $|r| \geq 1$ then the geometric series diverges.

% Problem 4
\begin{ex*} 
\begin {equation*}
0.\bar{9} = \frac{9}{10} \sum_{n=0}^{\infty}(\frac{1}{10})^n = \frac{\frac{9}{10}}{1-\frac{1}{10}} = 1
\end{equation*}
\end{ex*}

% Footer
\begin{center}
  \begin{footnotesize}
    Last updated: \today \\
  \end{footnotesize}
\end{center}
\end{document}
