\documentclass[11pt]{article}
\title{CSC 490: \LaTeX\ -- Week 3}
\author{Dor Rondel}

% packages
\usepackage{amsmath, amsthm, amssymb}
\usepackage{hyperref,color}
\usepackage{twoopt}
\usepackage[margin=1in]{geometry}

%  set the style for the theorems; optional
\theoremstyle{definition}

% new theorems
\newtheorem{prac}{Practice Problem}
\newtheorem{satz}{Satz}
\newtheorem{ph}[prac]{Practice Homework}  % follow prac counter

% new commands
\newcommand{\dmd}{\diamondsuit}
\newcommand{\rem}[1]{\pmod{#1}}
\newcommandtwoopt{\intZeroOne}[3][0][1]{\int_{#1}^{#2} #3}

% new environments
\newenvironment{gbl}{\par\color{green}\Large\textbf}{\par}

\begin{document}
\maketitle

% first problem
\section{Theorem Definition Practice}
\vline
\begin{prac} initial counter \end{prac}

\begin{satz} independent numbering \end{satz}

\begin{satz}
independent numbering \#2
\end{satz}

\begin{ph}
dependent numbering
\end{ph}

\begin{ph}
dependent numbering \#2
\end{ph}

\begin{satz}
independent numbering \#3
\end{satz}

\vline

% second problem
\section{New Command Practice}
\vspace{4px}
Instead of writing \verb \diamondsuite I can now write \verb \dmd to produce $\dmd$ \\
I can now use the \verb \rem to replicate \verb \pmod behavior, for example $12 \rem{8739}$ \\ 
To make a definite integral with I can write \verb \intZeroOne{x^5} to produce $\intZeroOne{x^5}$


\vline

% third problem
\section{New Environment Practice}
\begin{verbatim}
\newenvironment{gbl}{\Large\textbf\par\color{green}}{\par}
\end{verbatim}
I expect to see indented green bolded text with a larger font size than this comment. 

% fourth problem
\section{New Environment Usage}
\begin{gbl}
In this assignment I used three theorems, which do the following.
\begin{prac}\end{prac}
\begin{satz}\end{satz}
\begin{ph}\end{ph}
Additionally, I also made three new commands, to simplify the creation process, namely: \\
\verb \dmd $\dmd$ - which takes no arguments \\
\verb \rem \hspace{2px} $12 \rem{300}$ - which takes one argument \\
\verb \intZeroOne $\intZeroOne{\frac{\sin{x} + 6}{\tan{x^4}}}$ - which takes one argument and has 2 default arguments.
Finally, as you can see, the text is bold, large, and green, because it was within the gbl environment made just to do that.
\end{gbl}
\end{document}
