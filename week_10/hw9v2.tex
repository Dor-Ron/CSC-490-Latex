% Oswego beamer template
% Justin Ryan
% last changed: 3 Sep 2013

% feel free to make any improvements/changes you wish

\documentclass[9 pt]{beamer}
% \usepackage{default}
\usepackage{lmodern, natbib, hyperref}
\usepackage{graphicx, float, caption}
\usepackage{tikz}
\usepackage{animate}
\usepackage{amsmath,amsfonts,epsfig,pgf} % ,graphicx
\usepackage[utf8]{inputenc}
\usepackage[T1]{fontenc} % for the é

% choose your theme
\usetheme{Singapore} % Warsaw, Copenhagen, Darmstadt, Madrid, Singapore, etc...




\newcommand{\mancommon}[1]{%
\begin{tikzpicture}[ultra thick,scale=0.25]
\draw (0,0) -- (1,2) -- (2,0); %legs
\draw (1,2) -- (1,4); %trunk
\draw (1,4.7) circle (0.7cm); %head
#1
\end{tikzpicture}}
\newcommand{\mananimated}{\begin{animateinline}[autoplay,loop]{3}
\mancommon{\draw (-1,1) -- (1,3) -- (3,1);}
\newframe
\mancommon{\draw (-1,3) -- (3,3);}
\newframe
\mancommon{\draw (-1,1.5) -- (1,3) -- (3,2.5);}
\newframe
\mancommon{\draw (-1,3) -- (3,3);}
\end{animateinline}}





% custom SUNY Oswego color scheme
\definecolor{oswego}{rgb}{0.15,0.4,0.15}
\setbeamercolor*{palette primary}{fg=white, bg=oswego}
\setbeamercolor*{palette sidebar primary}{fg=black, bg=oswego}
\setbeamercolor{block title}{bg=black,fg=white} % bg=background, fg= foreground
\setbeamercolor{block body}{bg=oswego,fg=black} % bg=background, fg= foreground
\setbeamercolor{alerted text}{fg=black}
\usecolortheme[named={oswego}]{structure}
%% what about text in a bullet, e.g., "enumerate"? ...

% something I found to get alert blocks in the Oswego color scheme
\newenvironment<>{lakeblock}[1]{%
  \begin{actionenv}#2%
      \def\insertblocktitle{#1}%
      \par%
      \mode<presentation>{%
\setbeamercolor{block title}{fg=white,bg=black}
       \setbeamercolor{block body}{fg=white,bg=oswego}
            }%
      \usebeamertemplate{block begin}}
    {\par\usebeamertemplate{block end}\end{actionenv}}

% Oswego State logo on every page
\logo{\pgfputat{\pgfxy(-10.5,0.35)}{\pgfbox[center,center]
{\includegraphics[height=1.35cm]{oswego_logo}}}}

% now some stuff you can ignore, but might find useful

% commutative diagrams with XY-pic
\usepackage[all]{xy}
\SelectTips{cm}{}
% make \mathscr, TeX \cal, and Euler script *all* available
% (notice the new command names to avoid overlap and/or confusion)
\usepackage{mathrsfs}
\let\rscr=\mathscr % use \rscr{} for Ralph Smith fancy script
\let\mathscr=\relax
\let\mcal=\mathcal % use \mcal{} for TeX \cal script
\usepackage{eucal}
\let\escr=\mathcal % use \escr{} for Euler script
\let\mathcal=\relax
% a better "bar" thanks to Donald Arsenau -- see \pbar infra
\usepackage{accents}

% title page information 
\title[your title]{Paul Cézanne: The Process of Sight}
\author[Y.\ Name]{Dor Rondel}
\institute[SUNY Oswego]{Oswego State University\\ Department of Mathematics}
\date[Date]{3/27/18}

\begin{document}

\section{Sample Section}
\subsection{Sample Subsection}

\begin{frame}{}
 \titlepage
\end{frame}

% \begin{frame}{}
%
% \end{frame}

 \begin{frame}{About}
   \begin{center}
   \includegraphics[width=1in]{paul} \label{PC}
       \begin{itemize}
          \item \begin{center}Born in South France in January 19, 1839.
\end{center}
\transblindsvertical
          \item \begin{center}Self-taught (largely so)\end{center}
          \item \begin{center}Pissarro shared his knowledge\end{center}
          \item \begin{center}Visited Academie Suisse regularly\end{center}
          \item \begin{center}Met fellow painters Monet, and Renoir\end{center}
        \end{itemize}
    \end{center}
 \end{frame}

\begin{frame}{References}
\bibliographystyle{alpha}
\bibliography{references}
\end{frame}

\begin{frame}{Works \& Contributions}
\Large{Some of Paul Cezanne's \cite{cez} ~\ref{PC}  most famous pieces include:}
\begin{itemize}
\transblindshorizontal
  \item The Basket of Apples
  \pause
  \item Pyramid of Skulls
  \pause
  \item Mont Sainte-Victoire Series
  \pause
  \item The Card Players Series
  \pause
  \item The Bathers
\end{itemize}
\end{frame}

\begin{frame}{Long Day}
\begin{center}
\mananimated
\end{center}
\end{frame}

\begin{frame}{Transitions Table}
\begin{table}[H]
\centering
\caption{}\label{transitions}
\begin{tabular}{ll}
Command & Effect\\
\hline
\texttt{transblindsvertical} & vertical blind effect\\
\texttt{transblindshorizontal} & horizontal blind effect\\
\texttt{transboxin} & reveal from edges to center\\
\texttt{transboxout} & reveal from center to edges\\
\texttt{transdissolve} & slide dissolves away\\
\texttt{transglitter} & glitter sweeping in\\
\texttt{transsplitverticalin} & Sweeps two vertical lines in\\
\texttt{transsplitverticalout} & Sweeps two vertical lines out\\
\texttt{transsplithorizontalin} & Sweeps two horizontal lines in\\
\texttt{transsplithorizontalout}& Sweeps two horizontal lines out\\
\texttt{transwipe} & wiping away of the slide\\
\texttt{transduration{2}} & shows slide for 2 seconds\\
\end{tabular}
\end{table}
\end{frame}

\end{document}
